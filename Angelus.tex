% Afficher des recommendations concernant la syntaxe.
\RequirePackage[orthodox,l2tabu]{nag}
\RequirePackage{luatex85}
% Paramètres du document.
\documentclass[%
a5paper%                       Taille de page.
,11pt%                         Taille de police.
,DIV=15%                       Plus grand => des marges plus petites.
,titlepage=on%                 Faut-il une page de titre ?
,headings=optiontoheadandtoc%  Effet des paramètres optionnels de section.
,headings=small%
,parskip=false%
,openany%
]{scrbook}
\renewcommand*\partheademptypage{\thispagestyle{empty}}
\newcounter{facteur}\setcounter{facteur}{17}%%%%%%%%%%%%%% Paramètre pour la taille globale des partitions. par défaut~: 17
%\usepackage{geometry}
\usepackage{gredoc,mudoc,lyluatex}
\usepackage{pdfpages,transparent,array,ltablex}

%%%%%%%%%%%%%%%%%%%%%%% Paramètres variables %%%%%%%%%%%%%%%%%%%%%%%%%%%%%%%%%%%%%%%%%%%%%%%%%%
%%% Taille des partitions grégoriennes.                                                      %%
%\grechangedim{overhepisemalowshift}{.7mm}{scalable}
%\grechangedim{hepisemamiddleshift}{1.4mm}{scalable}
%\grechangedim{overhepisemahighshift}{2.1mm}{scalable}
%\grechangedim{vepisemahighshift}{2.1mm}{scalable}
%\grechangestafflinethickness{50} %%% epaisseur des lignes
\grechangestaffsize{\value{facteur}}%%%%% 
%%%%%%%%%%%%%%%%%%%%%%%%%%%%%%%%%%%%%%%%%%%%%%%%%%%%%%%%%%%%%%%%%%%%%%%%%%%%%%%%%%%%%%%%%%%%%%%
% Par souci de clarté, la définition des commandes est reportée dans un document annexe.

%\addtolength{\voffset}{2mm}\addtolength{\headsep}{-2mm}
%\setlength{\extrarowheight}{2mm}

\addto\captionsfrench{%
  \renewcommand{\indexname}{Index des chants}%
}

\pdfcompresslevel=9

\newcommand{\lieu}[1]{\hfill\linebreak[3]\hspace*{\stretch{1}}\nolinebreak\mbox{\emph{(#1)}}}

\newcommand{\commandement}[1]{\noindent\textbf{#1}}

\newcommand{\schola}[1]{}\newcommand{\foule}[1]{#1}
\providecommand{\dest}{foule}

\newcommand{\bgimage}[1]{%
\raisebox{-.45\paperheight}[0pt][0pt]{%
  \transparent{0.3}%
  \includegraphics[width=.7\paperwidth,height=.7\paperheight,keepaspectratio=true]{img/#1}%
  }%
}

\def\arraystretch{1.2}

\newcommand{\reponsegras}[2]{
\versio{\textbf{#1}}{\textbf{#2}}
}

\title{Jubilé de Notre-Dame de Fontpeyrine}
\date{}

\let\oldaddchap\addchap
\def\addchap#1{\oldaddchap{#1}\markright{Pèlerinage du Christ-Roi}}

\def\blindsection#1{\markright{#1}\addcontentsline{toc}{section}{#1}}
%%%%%%%%%%%%%%%%%%%%%%%%%%%%%%%%%%%%%%%%%%%%%%%%%%%%%%%%%%%%%%%%%%%%%%%%%%%%%%%%
%%%%%%%%%%%%%%%%%%%%%% Début du document %%%%%%%%%%%%%%%%%%%%%%%%%%%%%%%%%%%%%%%
%%%%%%%%%%%%%%%%%%%%%%%%%%%%%%%%%%%%%%%%%%%%%%%%%%%%%%%%%%%%%%%%%%%%%%%%%%%%%%%%
\begin{document}

\chapter{Angelus}

\reponsegras{\vb.\ Angelus Dómini nuntiávit Maríæ}{\vb.~L'ange du Seigneur fit l'annonce à Marie}
\versio{\rb.\ Et concépit de Spíritu Sancto}{\rb.\ Et elle a conçu du Saint-Esprit}
\reponsegras{Ave María…}{Je vous salue Marie…}
\reponsegras{\vb.~Ecce Ancílla Dómini}{\vb.~Voici la servante du Seigneur.}
\versio{\rb. Fiat mihi secúndum verb.um tuum}{\rb. qu’il me soit fait selon votre parole.}
\reponsegras{Ave María…}{Je vous salue Marie…}
\reponsegras{\vb.~Et Verb.um caro factum est}{\vb.~Et le Verb.e s'est fait chair}
\versio{\rb. Et habitávit in nobis}{\rb. Et il a habité parmi nous.}
\reponsegras{Ave María…}{Je vous salue Marie…}
\reponsegras{\vb.~Ora pro nobis, sancta Dei Génetrix.}{\vb.~Priez pour nous sainte Mère de Dieu}
\versio{\vb.\ Ut digni efficiámur promissiónibus Christi.}{\vb.\ Afin que nous soyons rendus dignes des promesses de Notre-Seigneur Jésus-Christ}
\smallskip
\versio{Orémus}{Prions}
\reponsegras{Grátiam tuam, quǽsumus, Dómine, méntibus nostris infúnde: ut, qui, Ángelo nuntiánte, Christi Fílii tui incarnatiónem cognóvimus; per passiónem eius et crucem, ad resurrectiónis glóriam perducámur. Per eúndem Christum Dóminum nóstrum.}
{Daignez, Seigneur, répandre votre Grâce dans nos âmes,
afin qu'ayant connu par la voix de l'Ange l'Incarnation de votre Fils Jésus-Christ,
nous arrivions, par les mérites de sa Passion et de sa Croix,
à la Gloire de la Résurrection.
Par le même Jésus-Christ, Votre Fils, Notre Seigneur. }
\versio{Amen.}{Ainsi soit-il.}

\end{document}