% Afficher des recommendations concernant la syntaxe.
\RequirePackage[orthodox,l2tabu]{nag}
\RequirePackage{luatex85}
% Paramètres du document.
\documentclass[%
a5paper%                       Taille de page.
,11pt%                         Taille de police.
,DIV=15%                       Plus grand => des marges plus petites.
,titlepage=on%                 Faut-il une page de titre ?
,headings=optiontoheadandtoc%  Effet des paramètres optionnels de section.
,headings=small%
,parskip=false%
,openany%
]{scrartcl}
%\renewcommand*\partheademptypage{\thispagestyle{empty}}
\newcounter{facteur}\setcounter{facteur}{17}%%%%%%%%%%%%%% Paramètre pour la taille globale des partitions. par défaut~: 17
%\usepackage{geometry}
\usepackage{gredoc,mudoc,lyluatex}
\usepackage{pdfpages,transparent,array,ltablex}

%%%%%%%%%%%%%%%%%%%%%%% Paramètres variables %%%%%%%%%%%%%%%%%%%%%%%%%%%%%%%%%%%%%%%%%%%%%%%%%%
%%% Taille des partitions grégoriennes.                                                      %%
%\grechangedim{overhepisemalowshift}{.7mm}{scalable}
%\grechangedim{hepisemamiddleshift}{1.4mm}{scalable}
%\grechangedim{overhepisemahighshift}{2.1mm}{scalable}
%\grechangedim{vepisemahighshift}{2.1mm}{scalable}
%\grechangestafflinethickness{50} %%% epaisseur des lignes
\grechangestaffsize{\value{facteur}}%%%%% 
%%%%%%%%%%%%%%%%%%%%%%%%%%%%%%%%%%%%%%%%%%%%%%%%%%%%%%%%%%%%%%%%%%%%%%%%%%%%%%%%%%%%%%%%%%%%%%%
% Par souci de clarté, la définition des commandes est reportée dans un document annexe.

%\addtolength{\voffset}{2mm}\addtolength{\headsep}{-2mm}
%\setlength{\extrarowheight}{2mm}


%\pdfcompresslevel=9

\def\arraystretch{1.2} %????

\newcommand{\reponsegras}[2]{
\versio{\textbf{#1}}{\textbf{#2}}}

%\let\oldaddchap\addchap
%\def\addchap#1{\oldaddchap{#1}\markright{Pèlerinage du Christ-Roi}}

%\def\blindsection#1{\markright{#1}\addcontentsline{toc}{section}{#1}}
%%%%%%%%%%%%%%%%%%%%%%%%%%%%%%%%%%%%%%%%%%%%%%%%%%%%%%%%%%%%%%%%%%%%%%%%%%%%%%%%
%%%%%%%%%%%%%%%%%%%%%% Début du document %%%%%%%%%%%%%%%%%%%%%%%%%%%%%%%%%%%%%%%
%%%%%%%%%%%%%%%%%%%%%%%%%%%%%%%%%%%%%%%%%%%%%%%%%%%%%%%%%%%%%%%%%%%%%%%%%%%%%%%%
\begin{document}

\title{Angelus}

\reponsegras{\vb.\ Angelus Dómini nuntiávit Maríæ,}{\vb.~L'ange du Seigneur fit l'annonce à Marie,}
\versio{\rb.\ Et concépit de Spíritu Sancto.}{\rb.\ Et elle a conçu du Saint-Esprit.}
\reponsegras{Ave María…}{Je vous salue Marie…}
\reponsegras{\vb.~Ecce Ancílla Dómini,}{\vb.~Voici la servante du Seigneur,}
\versio{\rb. Fiat mihi secúndum verbum tuum.}{\rb. qu’il me soit fait selon votre parole.}
\reponsegras{Ave María…}{Je vous salue Marie…}
\reponsegras{\vb.~Et Verbum caro factum est,}{\vb.~Et le Verbe s'est fait chair,}
\versio{\rb. Et habitávit in nobis.}{\rb. Et il a habité parmi nous.}
\reponsegras{Ave María…}{Je vous salue Marie…}
\reponsegras{\vb.~Ora pro nobis, sancta Dei Génetrix.}{\vb.~Priez pour nous sainte Mère de Dieu.}
\versio{\vb.\ Ut digni efficiámur promissiónibus Christi.}{\vb.\ Afin que nous soyons rendus dignes des promesses de Notre-Seigneur Jésus-Christ.}
%\smallskip
\versio{Orémus.}{Prions.}
\reponsegras{Grátiam tuam, quǽsumus Dómine, méntibus nostris infunde~:~:\\
ut, qui, Angelo nuntiánte, Christi Fílii tui incarnatiónem cognóvimus~;\\
per passiónem eius et crucem, ad resurrectiónis glóriam perducámur.\\
Per eúndem Christum Dóminum nóstrum.}
{Daignez, Seigneur, répandre votre Grâce dans nos âmes,\\
afin qu'ayant connu par la voix de l'Ange l'Incarnation de votre Fils Jésus-Christ,\\
nous arrivions, par les mérites de sa Passion et de sa Croix,\\
à la Gloire de la Résurrection.\\
Par le même Jésus-Christ, Votre Fils, Notre Seigneur. }
\versio{Amen.}{Ainsi soit-il.}

%\chapter{Angélus chanté}
\titre{L'Ange du Seigneur annonce}
\chanson[position=2col,numero=1]{ly/Angelus/LAngeDuSeigneur}

\titre{Par Dieu l'Archange fut envoyé}
\chanson[position=2col,numero=1]{ly/Angelus/ParDieuLArchange}

\titre{Voici que l'Ange Gabriel}
\chanson[position=2col,numero=1]{ly/Angelus/VoiciQueLAngeGabriel}

\newpage
\titre{Regina Cæli}
\cantus{Antienne}{ReginaCaeli-simple}{}{}
\reponsegras{\vb. Gaude et lætáre, Virgo María, allelúia.}{\vb. Réjouissez-vous et soyez dans l’allégresse, Vierge Marie, alléluia.}
\versio{Quia surréxit Dóminus vere, allelúia.}{Parce que le Seigneur est vraiment ressuscité, alléluia.}
\reponsegras{Orémus.}{Prions.}
\versio{Deus, qui per resurrectiónem Fílii tui, Dómini nostri Iesu Christi, mundum lætificáre dignátus es~: præsta, quǽsumus~; ut, per eius Genitrícem Vírginem Maríam, perpétuæ capiámus gáudia vitæ. Per eúndem Christum Dóminum nostrum.}
{O Dieu, qui avez daigné réjouir le monde par la résurrection de Jésus-Christ, votre Fils~; faites, nous vous en supplions, qu’aidés par sa Mère, la Vierge Marie, nous arrivions à la possession des joies de la vie éternelle. Par le même Christ notre Seigneur.}
\versio{Amen.}{Ainsi soit-il.}

\end{document}